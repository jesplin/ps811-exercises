% Options for packages loaded elsewhere
\PassOptionsToPackage{unicode}{hyperref}
\PassOptionsToPackage{hyphens}{url}
%
\documentclass[
  english,
  man]{apa6}
\usepackage{lmodern}
\usepackage{amssymb,amsmath}
\usepackage{ifxetex,ifluatex}
\ifnum 0\ifxetex 1\fi\ifluatex 1\fi=0 % if pdftex
  \usepackage[T1]{fontenc}
  \usepackage[utf8]{inputenc}
  \usepackage{textcomp} % provide euro and other symbols
\else % if luatex or xetex
  \usepackage{unicode-math}
  \defaultfontfeatures{Scale=MatchLowercase}
  \defaultfontfeatures[\rmfamily]{Ligatures=TeX,Scale=1}
\fi
% Use upquote if available, for straight quotes in verbatim environments
\IfFileExists{upquote.sty}{\usepackage{upquote}}{}
\IfFileExists{microtype.sty}{% use microtype if available
  \usepackage[]{microtype}
  \UseMicrotypeSet[protrusion]{basicmath} % disable protrusion for tt fonts
}{}
\makeatletter
\@ifundefined{KOMAClassName}{% if non-KOMA class
  \IfFileExists{parskip.sty}{%
    \usepackage{parskip}
  }{% else
    \setlength{\parindent}{0pt}
    \setlength{\parskip}{6pt plus 2pt minus 1pt}}
}{% if KOMA class
  \KOMAoptions{parskip=half}}
\makeatother
\usepackage{xcolor}
\IfFileExists{xurl.sty}{\usepackage{xurl}}{} % add URL line breaks if available
\IfFileExists{bookmark.sty}{\usepackage{bookmark}}{\usepackage{hyperref}}
\hypersetup{
  pdftitle={AP Seminar Week 3 Readings},
  pdfauthor={Jess Esplin1},
  pdflang={en-EN},
  pdfkeywords={keywords},
  hidelinks,
  pdfcreator={LaTeX via pandoc}}
\urlstyle{same} % disable monospaced font for URLs
\usepackage{graphicx,grffile}
\makeatletter
\def\maxwidth{\ifdim\Gin@nat@width>\linewidth\linewidth\else\Gin@nat@width\fi}
\def\maxheight{\ifdim\Gin@nat@height>\textheight\textheight\else\Gin@nat@height\fi}
\makeatother
% Scale images if necessary, so that they will not overflow the page
% margins by default, and it is still possible to overwrite the defaults
% using explicit options in \includegraphics[width, height, ...]{}
\setkeys{Gin}{width=\maxwidth,height=\maxheight,keepaspectratio}
% Set default figure placement to htbp
\makeatletter
\def\fps@figure{htbp}
\makeatother
\setlength{\emergencystretch}{3em} % prevent overfull lines
\providecommand{\tightlist}{%
  \setlength{\itemsep}{0pt}\setlength{\parskip}{0pt}}
\setcounter{secnumdepth}{-\maxdimen} % remove section numbering
% Make \paragraph and \subparagraph free-standing
\ifx\paragraph\undefined\else
  \let\oldparagraph\paragraph
  \renewcommand{\paragraph}[1]{\oldparagraph{#1}\mbox{}}
\fi
\ifx\subparagraph\undefined\else
  \let\oldsubparagraph\subparagraph
  \renewcommand{\subparagraph}[1]{\oldsubparagraph{#1}\mbox{}}
\fi
% Manuscript styling
\usepackage{upgreek}
\captionsetup{font=singlespacing,justification=justified}

% Table formatting
\usepackage{longtable}
\usepackage{lscape}
% \usepackage[counterclockwise]{rotating}   % Landscape page setup for large tables
\usepackage{multirow}		% Table styling
\usepackage{tabularx}		% Control Column width
\usepackage[flushleft]{threeparttable}	% Allows for three part tables with a specified notes section
\usepackage{threeparttablex}            % Lets threeparttable work with longtable

% Create new environments so endfloat can handle them
% \newenvironment{ltable}
%   {\begin{landscape}\begin{center}\begin{threeparttable}}
%   {\end{threeparttable}\end{center}\end{landscape}}
\newenvironment{lltable}{\begin{landscape}\begin{center}\begin{ThreePartTable}}{\end{ThreePartTable}\end{center}\end{landscape}}

% Enables adjusting longtable caption width to table width
% Solution found at http://golatex.de/longtable-mit-caption-so-breit-wie-die-tabelle-t15767.html
\makeatletter
\newcommand\LastLTentrywidth{1em}
\newlength\longtablewidth
\setlength{\longtablewidth}{1in}
\newcommand{\getlongtablewidth}{\begingroup \ifcsname LT@\roman{LT@tables}\endcsname \global\longtablewidth=0pt \renewcommand{\LT@entry}[2]{\global\advance\longtablewidth by ##2\relax\gdef\LastLTentrywidth{##2}}\@nameuse{LT@\roman{LT@tables}} \fi \endgroup}

% \setlength{\parindent}{0.5in}
% \setlength{\parskip}{0pt plus 0pt minus 0pt}

% \usepackage{etoolbox}
\makeatletter
\patchcmd{\HyOrg@maketitle}
  {\section{\normalfont\normalsize\abstractname}}
  {\section*{\normalfont\normalsize\abstractname}}
  {}{\typeout{Failed to patch abstract.}}
\patchcmd{\HyOrg@maketitle}
  {\section{\protect\normalfont{\@title}}}
  {\section*{\protect\normalfont{\@title}}}
  {}{\typeout{Failed to patch title.}}
\makeatother
\shorttitle{AP Seminar Week 3 Readings}
\keywords{keywords\newline\indent Word count: X}
\DeclareDelayedFloatFlavor{ThreePartTable}{table}
\DeclareDelayedFloatFlavor{lltable}{table}
\DeclareDelayedFloatFlavor*{longtable}{table}
\makeatletter
\renewcommand{\efloat@iwrite}[1]{\immediate\expandafter\protected@write\csname efloat@post#1\endcsname{}}
\makeatother
\usepackage{lineno}

\linenumbers
\usepackage{csquotes}
\ifxetex
  % Load polyglossia as late as possible: uses bidi with RTL langages (e.g. Hebrew, Arabic)
  \usepackage{polyglossia}
  \setmainlanguage[]{english}
\else
  \usepackage[shorthands=off,main=english]{babel}
\fi

\title{AP Seminar Week 3 Readings}
\author{Jess Esplin\textsuperscript{1}}
\date{}


\authornote{

Jess Esplin, Ph.D.~Student. Department of Political Science at the University of Wisconsin-Madison.

This document is an exercise for PS811-Introduction to R and reviews each reading assigned for Week 3 of the Fall 2020 AP Seminar class, PS904.

Correspondence concerning this article should be addressed to Jess Esplin, Dept. of Political Science, UW-Madison. E-mail: \href{mailto:jesplin@wisc.com}{\nolinkurl{jesplin@wisc.com}}

}

\affiliation{\vspace{0.5cm}\textsuperscript{1} University of Wisconsin-Madison}

\begin{document}
\maketitle

\hypertarget{logistics}{%
\section{Logistics}\label{logistics}}

I used R (Version 4.0.2; R Core Team, 2020) and the R-package \emph{papaja} (Version 0.1.0.9997; Aust \& Barth, 2020) for this exercise.

\hypertarget{assigned-readings-for-week-3-of-ap-seminar}{%
\section{Assigned Readings for Week 3 of AP Seminar}\label{assigned-readings-for-week-3-of-ap-seminar}}

The assigned readings for Week 3 were Zaller (1992) , Zaller (2012), and Kinder (1998). Each work discusses important aspects of mass public opinion.

Zaller (1992) incorporates existing scholarship in public opinion and psychology to build a comprehensive model demonstrating how individuals form political opinions in the form of a Bayesian updating model, which we call the RAS model for Receive, Accept, Sample. The model is then scaled up to explain how mass public opinion is largely shaped by exposure to elite discourse on issues through each individuals' political awareness, generally facilitated by the media. Zaller tests this model using NES survey data, applying the RAS to a broad range of issues, including racial equality, the Vietnam War, and presidential approval (Zaller, 1992).

In 2012, Zaller returns to the topic to address criticisms of the RAS model and update it to account for group interest voters and political parties as political organizers. He explains that ideologies are packaged together and sold to voters by political parties in order to create coalitions, appease interest groups, and fundraise. This explains Converse (2006) 's findings that the majority of the public does not have an underlying belief structure. He again uses NES data to test this updated version of the RAS model (Zaller, 2012).

Kinder (1998) reviews the literature on public opinion for The Handbook of Social Psychology. He addresses many topics within public opinion, including its definition, Democratic citzenship, how opinion is formed, and the importance of social context. Notably, Kinder posits challenges of democracy can be resolved through \enquote{miracles of aggregation} - individuals may be ignorant but the public is made wiser through shared knowledge and he proposes a more central focus on ideas borrowed from psychology of framing, group consciousness, and identity to explain political behavior. Kinder's book chapter serves as a meta-analysis to synthesize and focus scholarship within public opinion (Kinder, 1998).

\newpage

\hypertarget{references}{%
\section{References}\label{references}}

\begingroup
\setlength{\parindent}{-0.5in}
\setlength{\leftskip}{0.5in}

\hypertarget{refs}{}
\leavevmode\hypertarget{ref-R-papaja}{}%
Aust, F., \& Barth, M. (2020). \emph{papaja: Create APA manuscripts with R Markdown}. Manual.

\leavevmode\hypertarget{ref-converseNatureBeliefSystems2006}{}%
Converse, P. E. (2006). The nature of belief systems in mass publics (1964). \emph{Critical Review}, \emph{18}(1-3), 1--74. \url{https://doi.org/10.1080/08913810608443650}

\leavevmode\hypertarget{ref-kinderOpinionActionRealm1998}{}%
Kinder, D. (1998). Opinion and action in the realm of politics. \emph{undefined}. /paper/Opinion-and-action-in-the-realm-of-politics.-Kinder/ff67b65efb0206133e54b7af3bd28bb1112dc7c1.

\leavevmode\hypertarget{ref-R-base}{}%
R Core Team. (2020). \emph{R: A language and environment for statistical computing}. Manual, Vienna, Austria: R Foundation for Statistical Computing.

\leavevmode\hypertarget{ref-zallerNatureOriginsMass1992}{}%
Zaller, J. (1992). \emph{The nature and origins of mass opinion}. Cambridge {[}England{]} ; New York, NY, USA: Cambridge University Press.

\leavevmode\hypertarget{ref-zallerWhatNatureOrigins2012}{}%
Zaller, J. (2012). What Nature and Origins Leaves Out. \emph{Critical Review}, \emph{24}(4), 569--642. \url{https://doi.org/10.1080/08913811.2012.807648}

\endgroup


\end{document}
